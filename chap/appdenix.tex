\newpage
%\appendix
\section{The detail proof of Theorem~\ref{lemma:warmnum1} and Theorem~\ref{lemma:warmnum2}}
\label{appendixa}
In this section, we describe the proof details of Theorem~\ref{lemma:warmnum1} and Theorem~\ref{lemma:warmnum2}.

For the following calculation, $0 \leq P_{a}^{\mathfrak{1}} \leq 1$, $0 \leq P_{a}^{\mathfrak{2}} \leq 1$, $0 \leq P_{a}^{\mathfrak{2}\mathcal{L}} \leq 1$, $0 \leq P_{a}^{\mathfrak{2}\mathcal{M}} \leq 1$, $0 \leq P_{warm}^{\mathcal{L}} \leq 1$, $0 \leq P_{warm}^{\mathcal{M}} \leq 1$
 and $0 \leq P_{evict} \leq 1$.

\subsection{The proof of Theorem~\ref{lemma:warmnum1}}
To prove the Theorem~\ref{lemma:warmnum1}, we need to compare $E(T^{\mathcal{L}})$ with $E(T^{\mathcal{C}})$.
So we get the Formula~\ref{equ:tpc1}, and estimate if it is larger than $0$.
By combing Equation~\ref{equ:pic} and ~\ref{equ:pil}, we get Equation~\ref{equ:ypwarm1} and \ref{equ:detail1}.

%\begin{footnotesize}
\begin{equation}
\label{equ:ypwarm1}
\begin{split}
%\footnotesize
%\scriptsize
y(P_{warm}^{\mathcal{L}}) =& -P_{a}^{\mathfrak{2}}(P_{a}^{\mathfrak{2}\mathcal{L}}-P_{evict}(P_{a}^{\mathfrak{2}\mathcal{L}}+P_{a}^{\mathfrak{1}}-P_{a}^{\mathfrak{3}\mathcal{L}}))*P_{warm}^{\mathcal{L}} \\
& +  P_{evict}P_{a}^{\mathfrak{2}\mathcal{L}}(1-P_{a}^{\mathfrak{1}})+P_{a}^{\mathfrak{2}\mathcal{L}}P_{a}^{\mathfrak{2}}\\
& - P_{evict}P_{a}^{\mathfrak{2}}(1+P_{a}^{\mathfrak{2}\mathcal{L}}-P_{a}^{\mathfrak{3}\mathcal{L}}) \,
\end{split}\end{equation}

\begin{equation}
\label{equ:detail1}
\begin{split}
E(T^{\mathcal{L}}) - E(T^{\mathcal{C}}) = &\frac{y(P_{warm}^{\mathcal{L}})P_{evict}T_{delay}}{A_{L}*B_{L}}
\end{split}\end{equation}
%\end{footnotesize}
where
\begin{equation*}
\begin{split}
&A_{L}=P_{evict}(1+P_{a}^{\mathfrak{2}\mathcal{L}}-P_{a}^{\mathfrak{3}\mathcal{L}})+(1-P_{evict})P_{a}^{\mathfrak{2}\mathcal{L}}P_{warm}^{\mathcal{L}}  \\
&\ \ \ \ \ \ \ +(P_{a}^{\mathfrak{3}\mathcal{L}}-P_{a}^{\mathfrak{1}})P_{evict}P_{warm}^{\mathcal{L}} \\
&B_{L}=P_{a}^{\mathfrak{2}}+P_{evict}(1-P_{a}^{\mathfrak{1}})
\end{split}\end{equation*}
In the Markov state transition diagram,
\begin{equation*}
\begin{split}
&P_{a}^{\mathfrak{3}\mathcal{L}}=P_{evict}^{\mathfrak{3}\mathcal{L}}P_{a}^{\mathfrak{31}\mathcal{L}}+(1-P_{evict}^{\mathfrak{3}\mathcal{L}})P_{a}^{\mathfrak{32}\mathcal{L}} .
\end{split}\end{equation*}
It can be seen from analysis that $P_{a}^{\mathfrak{31}\mathcal{L}} > P_{a}^{\mathfrak{1}}$ and $P_{a}^{\mathfrak{32}\mathcal{L}} > P_{a}^{\mathfrak{1}}$.
So we can get $P_{a}^{\mathfrak{3}\mathcal{L}} > P_{a}^{\mathfrak{1}}$.
In this way,
the function $y(P_{warm}^{\mathcal{L}})$ is a monotony decrease function. When $P_{warm}^{\mathcal{L}} = 1$, it gets the minimum value.
At this point, $P_{a}^{\mathfrak{2}\mathcal{L}} = P_{a}^{\mathfrak{2}}$ and $y(P_{warm}^{\mathcal{L}}) = 0$.
Also $A_{L}$ and $B_{L}$ are greater than zero.
Therefore $E(T^{\mathcal{L}}) - E(T^{\mathcal{C}})\geq 0$.
That means the extra time introduced by less {\vwarm} strategy is always larger than conditional {\vwarm} strategy.

\subsection{The proof of Theorem~\ref{lemma:warmnum2}}
To prove the Theorem~\ref{lemma:warmnum2}, we need to compare $E(T^{\mathcal{M}})$ with $E(T^{\mathcal{C}})$.
So we get the Formula~\ref{equ:tpc2}, and estimate if it is larger than $0$.

First, we find that $P_{a}^{\mathfrak{2}} = P_{a}^{\mathfrak{2}\mathcal{M}}$.
Then by combing Equation~\ref{equ:pic} and ~\ref{equ:pim}, we get Equation~\ref{equ:ypwarm2} and \ref{equ:detail2}.

%For the calculation, $0 \leq P_{a1} \leq 1$, $0 \leq P_{a2} \leq 1$, $0 \leq P_{a2}^{'} \leq 1$, $0 \leq P_{a2}^{''} \leq 1$, $0 \leq P_{Lwarm} \leq 1$, $0 \leq P_{Mwarm} \leq 1$
% and $0 \leq P_{evict} \leq 1$.

%\begin{footnotesize}
\begin{equation}
\label{equ:ypwarm2}
\begin{split}
%\footnotesize
%\scriptsize
y(P_{Mwarm}) = & A_{M}C_{M}P_{Mwarm}^2+B_{M}C_{M}P_{Mwarm} \\
&-gD_{M}P_{Mwarm} \,
\end{split}\end{equation}

\begin{equation}
\label{equ:detail2}
\begin{split}
E(T^{M}) - E(T^{C}) = &\frac{y(P_{Mwarm})T_{warm}}{E_{M}*F_{M}}
\end{split}\end{equation}
%\end{footnotesize}
where
\begin{equation*}
\begin{split}
&A_{M}= Mg(1-P_{a2})\\
&B_{M}=MgP_{a2}+P_{evict}(1-P_{a1})\\
&C_{M}=\frac{P_{a2}+P_{evict}P_{a2}+P_{evict}(1-P_{a1})}{P_{evict}(1-P_{a1}}\\
&D_{M}=-P_{evict}P_{a1}+P_{evict}P_{a2}+P_{a2}\\
&E_{M}=(1+P_{evict})(1-(1-P_{Mwarm})(1-P_{a2})) \\& \ \ \ \ \ \ \ \ +P_{evict}(1-P_{Mwarm})(1-P_{a1})\\
&F_{M}=P_{a2}+P_{evict}P_{a2}+P_{evict}(1-P_{a1})
\end{split}\end{equation*}

Because $E_{M}$ and $F_{M}$ are greater than zero, to make $E(T^{M}) - E(T^{C}) \geq 0$, $y(P_{Mwarm})$ should be equal or greater than zero.
Therefore we can get $ A_{M}C_{M}P_{Mwarm}+B_{M}C_{M}-gD_{M} \geq 0$.
If this inequality holds for all $P_{Mwarm} \in [0,1]$, it should be satisfied that $B_{M}C_{M}-gD_{M} \geq 0$.
Then we get the following inequality.

\begin{equation}
\label{equ:gcondition}
\begin{split}
&((P_{evict}P_{a2}+P_{a2}-P_{evict}P_{a1})(P_{evict}(1-P_{a1})-MP_{a2}) \\
&-MP_{evict}P_{a2})g \leq \\
&P_{evict}(1-P_{a1})(P_{evict}P_{a2}+P_{a2}+P_{evict}(1-P_{a1}))
\end{split}\end{equation}

We denote $(P_{evict}P_{a2}+P_{a2}-P_{evict}P_{a1})(P_{evict}(1-P_{a1})-MP_{a2})-MP_{evict}P_{a2}$ as $GDenominator$.
\begin{enumerate}
\item
If $GDenominator\leq0$,
then
because the right of the inequality is greater than zero, the inequality holds.

In this case, we regard $GDenominator$ as a function of $P_{evict}$ and get:
\begin{equation}
%\label{equ:detail2}
\begin{split}
&f(P_{evict})=(P_{a2}-P_{a1})(1-P_{a1})P_{evict}^2 \\
&+((1-M)P_{a2}-(1-M)P_{a1}P_{a2}-MP_{a2}^2)P_{evict}\\
&-MP_{a2}^2
\end{split}\end{equation}
It is an quadratic function and when $P_{evict} = 0$, $f(P_{evict})<0$.
So
(1) when $P_{a2}>P_{a1}$ and $P_{evict}< \min(1,P_{rootr})$,
    $f(P_{evict}) \leq0$;($P_{rootr}$ is the greater root of equation $f(P_{evict})=0$.)
(2) when $P_{a2}=P_{a1}$ and $P_{evict}< \min(1,\frac{M*P_{a1}}{1-M-P_{a1}})$, $f(P_{evict}) \leq0$; ($1-M-P_{a1}>0$, if not,$f(P_{evict}) \leq0$ holds.)
(3) when $P_{a2}<P_{a1}$, the symmetry axis of $f(P_{evict})$ is $P_{evict} = P_{e\_axis}$, where $P_{e\_axis}$ is denoted in Equation~\ref{equ:eaxis}.
\begin{equation}
\label{equ:eaxis}
\begin{split}
P_{e\_axis} = &-\frac{P_{a2}(1-M-P_{a1}-MP_{a2}+MP_{a1})}{2(P_{a2}-P_{a1})(1-P_{a1})}
\end{split}
\end{equation}
If $P_{e\_axis} \leq 0$, i.e. $P_{a2}\geq \frac{(1-M)(1-P_{a1})}{M}$, $f(P_{evict}) \leq0$ holds.
Otherwise, $P_{e\_axis} > 0$, i.e. $P_{a2} < \frac{(1-M)(1-P_{a1})}{M}$. In this case, if $P_{rootl}$ exists, $P_{evict}< min(1,P_{rootl})$ and then $f(P_{evict}) \leq0$. If $P_{rootl}$ does not exist, $f(P_{evict}) \leq0$ holds. ($P_{rootl}$ is the smaller root of equation $f(P_{evict})=0$.)

\item
If $GDenominator > 0$,
then
\begin{equation}
\begin{split}
g \leq \frac{P_{evict}(1-P_{a1})(P_{a2}+P_{evict}P_{a2}+P_{evict}(1-P_{a1}))}{GDenominator}
\end{split}
\end{equation}

We denote $\frac{P_{evict}(1-P_{a1})(P_{a2}+P_{evict}P_{a2}+P_{evict}(1-P_{a1}))}{GDenominator}$ as $Gexpression$.
First, we regard $Gexpression$ as the function of $P_{a1}$ and calculate the derivative on $P_{a1}$, and get:
\begin{equation}
\begin{split}
\frac{d(Gexpression)}{d(P_{a1})} = &P_{evict}^{2}(MP_{a2}+P_{evict})(1-P_{a1})^2\\
&+2MP_{evict}(1+P_{evict})P_{a2}^{2}(1-P_{a1})\\
&+MP_{a2}^{3}(1+P_{evict})^2
\end{split}
\end{equation}
Therefore, $Gexpression$ is a monotony increase function of $P_{a1}$.

Second, we regard $Gexpression$ as the function of $P_{a2}$ and calculate the derivative on $P_{a2}$, and get:
\begin{equation}
\begin{split}
\frac{d(Gexpression)}{d(P_{a2})} = & P_{evict}\bar{P}_{a1}\{M(1+P_{evict})^{2}P_{a2}^2\\
&+2MP_{evict}\bar{P}_{a1}(1+P_{evict})P_{a2} \\
&+MP_{evict}^{2}\bar{P}_{a1}^2 \\
&-P_{evict}^{2}\bar{P}_{a1}(1+P_{evict})\}
\end{split}
\end{equation}
where
$$
\bar{P}_{a1} = 1 - P_{a1}
$$

Denote $P_{a2r}$ as the greater root of equation $\frac{d(Gexpression)}{d(P_{a2})}=0$.
$M(1+P_{evict})^{2} > 0$ and when $P_{a2} = 0$, so $\frac{d(Gexpression)}{d(P_{a2})} < 0$,
$P_{a2r} > 0$.
\begin{equation}
\begin{split}
P_{a2r}=\frac{P_{evict}\sqrt{\frac{(1+P_{evict})\bar{P}_{a1}}{M}}-P_{evict}\bar{P}_{a1}}{1+P_{evict}}
\end{split}
\end{equation}

Therefore, if $P_{a2r} \geq 1$, then $\frac{d(Gexpression)}{d(P_{a2})} < 0$ and $Gexpression$ is a monotony decrease function of $P_{a2} \in [0,1]$.
And if $P_{a2r} < 1$, then $Gexpression$ first decreases then increase as the function of $P_{a2} \in [0,1]$.

Let $P_{a2r} = 1$, and we get:
\begin{equation}
\begin{split}
\bar{P}_{a1}P_{evict}^3& + (\bar{P}_{a1}-M(1+\bar{P}_{a1})^2)P_{evict}^2\\
& - 2M(\bar{P}_{a1}+1)P_{evict}-M = 0
\end{split}
\end{equation}
It is a simple cubic equation on $P_{evict}$. We denote $P_{e\_r}$ as its greatest real root.
If $P_{e\_r} < 1$,
then, when $P_{evict} \geq P_{e\_r}$, $P_{a2r} \geq 1$. When $P_{evict} < P_{e\_r}$, $P_{a2r} < 1$.
Otherwise, $P_{a2r} < 1$ holds.
In general,if $P_{e\_r} < 1$, when $P_{evict} \geq P_{e\_r}$, $Gexpression$ is a monotony decrease function of $P_{a2} \in [0,1]$.
 When $P_{evict} < P_{e\_r}$, $Gexpression$ first decreases then increase as the function of $P_{a2} \in [0,1]$.
 And if not, $Gexpression$ first decreases then increase as the function of $P_{a2} \in [0,1]$.

 Third, we regard $Gexpression$ as the function of $P_{evict}$ and calculate the derivative on $P_{evict}$, and get:
\begin{equation}
\begin{split}
\frac{d(Gexpression)}{d(P_{evict})} &= P_{a2}(\bar{P}_{a1}-M(\bar{P}_{a1}+P_{a2})^2)P_{evict}^2\\
&-2MP_{a2}^{2}(P_{a2}+\bar{P}_{a1})P_{evict}\\
&-MP_{a2}^3
\end{split}
\end{equation}
If $P_{a2}=\sqrt{\frac{\bar{P}_{a1}}{M}}-\bar{P}_{a1}$, $\frac{d(Gexpression)}{d(P_{evict})} < 0$. Function $Gexpression$ is a monotony decrease function on $P_{evict}$.
Otherwise, $\frac{d(Gexpression)}{d(P_{evict})}$ is a simple cubic equation on $P_{evict}$.
The symmetry axis of the function $\frac{d(Gexpression)}{d(P_{evict})}$ is:
\begin{equation}
\begin{split}
P_{e\_axis}=\frac{MP_{a2}(\bar{P}_{a1}+P_{a2})}{\bar{P}_{a1}-M(\bar{P}_{a1}+P_{a2})^2}
\end{split}
\end{equation}
(1) If $P_{a2}< \sqrt{\frac{\bar{P}_{a1}}{M}}-\bar{P}_{a1}$,
 then $P_{e\_axis} > 0$.

 When $P_{evict} = 0$, $\frac{d(Gexpression)}{d(P_{evict})} = -MP_{a2}^3 < 0$.
 When $P_{evict} = 1$,
 $$
 \frac{d(Gexpression)}{d(P_{evict})} = P_{a2}(\bar{P}_{a1}-M(\bar{P}_{a1}+2P_{a2})^2)
 $$
So when $P_{a2}< \frac{1}{2}(\sqrt{\frac{\bar{P}_{a1}}{M}}-\bar{P}_{a1})$, $\frac{d(Gexpression)}{d(P_{evict})} > 0$.
 In this case, $Gexpression$ first decreases then increase as the function of $P_{evict} \in [0,1]$.
when $P_{a2}\geq \frac{1}{2}(\sqrt{\frac{\bar{P}_{a1}}{M}}-\bar{P}_{a1})$, $\frac{d(Gexpression)}{d(P_{evict})} < 0$.
 In this case, $Gexpression$ is a  a monotony decrease function of $P_{evict} \in [0,1]$.
(2) If $P_{a2}> \sqrt{\frac{\bar{P}_{a1}}{M}}-\bar{P}_{a1}$,
then $P_{e\_axis} < 0$.
 When $P_{evict} = 0$, $\frac{d(Gexpression)}{d(P_{evict})} = -MP_{a2}^3 < 0$.
So in this case, $Gexpression$ is a  a monotony decrease function of $P_{evict} \in [0,1]$.

In general, when $P_{a2}< \frac{1}{2}(\sqrt{\frac{\bar{P}_{a1}}{M}}-\bar{P}_{a1})$, $Gexpression$ first decreases then increase as the function of $P_{evict} \in [0,1]$; when $P_{a2}\geq \frac{1}{2}(\sqrt{\frac{\bar{P}_{a1}}{M}}-\bar{P}_{a1})$, $Gexpression$ is a  a monotony decrease function of $P_{evict} \in [0,1]$.

In a conclusion, if $GDenominator > 0$,
$Gexpression$ is a monotony increase function of $P_{a1}$;
 when $P_{evict} < \min(P_{e\_r},1)$, $Gexpression$ first decreases then increase as the function of $P_{a2} \in [0,1]$;
 if $P_{e\_r} < 1$, when $P_{evict} \geq P_{e\_r}$, $Gexpression$ is a monotony decrease function of $P_{a2} \in [0,1]$;
when $P_{a2}< \frac{1}{2}(\sqrt{\frac{\bar{P}_{a1}}{M}}-\bar{P}_{a1})$, $Gexpression$ first decreases then increase as the function of $P_{evict} \in [0,1]$; when $P_{a2}\geq \frac{1}{2}(\sqrt{\frac{\bar{P}_{a1}}{M}}-\bar{P}_{a1})$, $Gexpression$ is a  a monotony decrease function of $P_{evict} \in [0,1]$.

Therefore, the minimum value of $Gexpression$ is get when $P_{a1}$ achieves the minimum value, $P_{a2}$ and $P_{evict}$ achieve the maximum value.
\end{enumerate}

From the above, we can make the following conclusions:
\begin{enumerate}
  \item
  (1) when $P_{a2}>P_{a1}$ and $P_{evict}< \min(1,P_{rootr})$,
  (2) when $P_{a2}=P_{a1}$ and $P_{evict}< \min(1,\frac{M*P_{a1}}{1-M-P_{a1}})$ (if $1-M-P_{a1}>0$),
  (3) when $P_{a2}<P_{a1}$, $P_{a2} < \frac{(1-M)(1-P_{a1})}{M}$ and $P_{evict}< min(1,P_{rootl})$ (if $P_{rootl}$ exists)
or $P_{a2}\geq \frac{(1-M)(1-P_{a1})}{M}$,
  we have $GDenominator\leq0$ and the Inequality~\ref{equ:gcondition} holds.

  \item
  when conditions are opposite, i.e. $GDenominator>0$,  if $g$ is smaller than the minimum value of $Gexpression$, the Inequality~\ref{equ:gcondition} holds.

\end{enumerate}
At this point, $E(T^{M}) - E(T^{C})\geq 0$ and conditional {\vwarm} strategy has better performance than the less {\vwarm} strategy.




%The symmetry axis of $y(P_{warm})$ is $P_{warm} = P_{a}$, where $P_{a}$ is denoted in Equation~\ref{equ:pa}.
%%\begin{footnotesize}
%\begin{equation}
%\label{equ:pa}
%\begin{split}
%P_{a} = &\frac{1}{2(P_{evict}+P_{delay})}\{k*P_{evict}P_{delay}\\
%& + P_{evict}^{2}P_{delay} \\
%&+ P_{evict}P_{delay}^{2} - P_{evict}^{2} \\
%& - P_{evict}P_{delay}\} \ .
%\end{split}
%\end{equation}
%%\end{footnotesize}
%Equation~\ref{equ:pal0} holds when $0.8 \leq P_{delay} \leq 1$, $0 \leq P_{evict} \leq 1$,  which means $P_{a}  \geq 0$.
%%\begin{footnotesize}
%  \begin{equation}
%  \label{equ:pal0}
%  \begin{split}
%     k  > 1 > & \frac{P_{evict}^{2} + P_{evict}P_{delay} - P_{evict}P_{delay}^{2} - P_{evict}^{2}P_{delay}}{P_{evict}P_{delay}} \\
%     &=1+ \frac{P_{evict}}{P_{delay}} - P_{delay} - P_{evict} \ .
%  \end{split}
%  \end{equation}
% % \end{footnotesize}
%  %We make $k_{1} = \frac{P_{evict}^{2} + P_{evict}P_{delay} - P_{evict}P_{delay}^{2} - P_{evict}^{2}P_{delay}}{P_{evict}P_{delay}}$.
%  %After simplified, $k_{1} = \frac{P_{evict}}{P_{delay}} - P_{delay} + 1 - P_{evict}$.
%  %$k_{1}$ is a decreasing function of $P_{delay}$, the maximum of $k_{1}$ is $0.25P_{evict}+0.2$ when $P_{delay} = 0.8$. The maximum value is less than 1, but $k$ must be larger than $1$. So this condition is false.
% %means $k<1$ which conflicts with the precondition $k>1$.
%
%
%
%\begin{enumerate}
%
%
%  \item If $P_{a} > 1$ (i.e., Equation~\ref{equ:pal1} holds), the minimum value of $y(P_{warm})$ is $y(1)$.
%  When $k \epsilon (k_{1}, k_{0}]$, $y(P_{warm}) \geq 0$ holds.
%  %\begin{footnotesize}
%    \begin{equation}
%    \label{equ:pal1}
%     k_{1} = \frac{(P_{evict}+P_{delay})(2+P_{evict}-P_{delay}P_{evict})}{P_{evict}P_{delay}} \ .
%  \end{equation}
%  %\end{footnotesize}
%  %In this condition,.
% % \begin{equation}
%  %  \scriptsize
%  %  y(1) = (P_{delay} - 1)P_{evict}P_{delay}k + (1 + P_{evict}- P_{evict}P_{delay})(P_{evict}+P_{delay}) \ .
% % \end{equation}
%  %make $y(1) \geq 0$, so all $y(P_{warm}) \geq 0$ when $0 \leq P_{warm} \leq 1$ . So:
%  %\begin{footnotesize}
%  \begin{equation}
%   k_{0} = \frac{(1 + P_{evict}- P_{evict}P_{delay})(P_{evict}+P_{delay})}{(1 - P_{delay})P_{evict}P_{delay}}\ .
%  \end{equation}
%  %\end{footnotesize}
%  %We mark $k_{2} = \frac{(P_{evict}+P_{delay})(2+P_{evict}-P_{delay}P_{evict})}{P_{evict}P_{delay}}$ and \par $k_{0} = \frac{(1 + P_{evict}- P_{evict}P_{delay})(P_{evict}+P_{delay})}{(1 - P_{delay})P_{evict}P_{delay}}$ ,\par so when
%  %$k_{2} < k \leq k_{0},$
% % $$y(P_{warm}) \geq 0$$.\par
%
% % \ \par
%
%  \item If $0 \leq P_{a} \leq 1,$ (i.e., $1 < k \leq k_{1}$), the minimum value of $y(P_{warm})$ is $y(P_{a})$.
%  %\begin{scriptsize}
%  \begin{equation}\begin{split}
%    %\scriptsize
%    &y(P_{a}) = \frac{1}{4(P_{evict}+P_{delay})}\{-P_{evict}^{2}P_{delay}^{2}k^{2}\\
%    &+ 2P_{evict}^{2}P_{delay}(2P_{delay}+P_{evict}-P_{evict}P_{delay})k\\
%    &+ 2P_{evict}P_{delay}^{2}(2P_{delay}+P_{evict}-P_{evict}P_{delay})k\\
%    & - P_{evict}^{2}(P_{evict}+P_{delay})^{2}(1-P_{delay})^{2}\}.
%  \end{split}\end{equation}
%%\end{scriptsize}
% $y(P_{a})$ is a function of $k$ (denoted as $f(k)$). As $-P_{evict}^{2}P_{delay}^{2} < 0$, the minimum of $f(k)$ is either $f(1)$ or $f(k_{1})$, for $1 < k \leq k_{1}$. When $P_{delay} \in [0.8,1]$, $f(1)$ and  $f(k_{2})$ is larger than $0$, which means $y(P_{a}) > 0$
%
%  % So if $f(1)>0 and f(k_{2})>0$, $f(k)>0 in k \in (1,k_{2}]$.
%  %\begin{scriptsize}
%  \begin{equation}
%  \label{equ:f1}
%  \begin{split}
%    %\scriptsize
%    f(1) =& \frac{1}{4(P_{evict}+P_{delay})}\{-(1-P_{delay})^{2}P_{evict}^{4}\\ & + 2P_{evict}^{3}P_{delay}^{2} + 4P_{evict}P_{delay}^{3} \\ &+ P_{evict}^{2}P_{delay}^{2}(4-P_{delay}^{2} - 2P_{evict}P_{delay})\}.
%  \end{split}\end{equation}%\end{scriptsize}
% % So, $f(1) > 0$ is true when $P_{delay} \in [0.8,1]$.
%
%  %\begin{footnotesize}
%  \begin{equation}
%  \label{equ:fk1}
%    \raggedleft
%  \begin{split}
%    f(k_{1}) = &(P_{evict}+P_{delay})\{2P_{delay}- 1\\ & + P_{evict}P_{delay}(1 - P_{delay})\}.
%  \end{split}\end{equation}
%  %\end{footnotesize}
%  %So, $f(k_{2}) > 0$ is true when $P_{delay} \in [0.8,1]$.\par
%
% % \ \par
%
%  %Therefore, when $1< k \leq k_{2}$, $y(P_{a}) > 0$.
%
%  %So, $y(P_{warm}) > 0$.
%
%\end{enumerate}
%%So, we can get that when $1< k \leq k_{0}$, $y(P_{warm}) \geq 0$.
%From the above analysis, we find that when $1< k \leq k_{0}$, $E(T^{p}) \geq E(T^{c})$.
%%I.e. when $1< k \leq k_{0}$, $T^{p} \geq T^{C}$

\section{The $P_{delay}$ in different AES key lengthes and different AES implementations }
\label{appendixb}
We assume the size of a cache line is $C$ Byte. $P_{delay}$ represents that the probability of performing delay operation with $N$ cache line size of lookup tables not in the cache. So when the AES implementation uses 4.25KB lookup tables, the $P_{delay}$ for AES-128, AES-192 and AES-256 are as follows respectively:
\begin{align}
%\begin{flalign}
%\begin{split}
    P_{delay}^{128} = &1-   \nonumber \\
    \frac{1}{\binom{4352/C}{N}}&\sum_{x_4=x_{4b}}^{x_{4e}}{\sum_{x_0=x_{0b}}^{x_{0e}}{\sum_{x_1=x_{1b}}^{x_{1e}}{ \sum_{x_2=x_{2b}}^{x_{2e}}{f(x_4)\prod_{i=0}^{3}{g(x_i,36)}}}}},
%\end{split}
%\end{flalign}
\end{align}
%for AES-192 is
\begin{align}
    %\scriptsize
    P_{delay}^{192} = &1-  \nonumber \\
    \frac{1}{\binom{4352/C}{N}}&\sum_{x_4=x_{4b}}^{x_{4e}}{\sum_{x_0=x_{0b}}^{x_{0e}}{\sum_{x_1=x_{1b}}^{x_{1e}}{ \sum_{x_2=x_{2b}}^{x_{2e}}{f(x_4)\prod_{i=0}^{3}{g(x_i,44)}}}}},
\end{align}
%for AES-256 is
\begin{align}
    %\scriptsize
    P_{delay}^{256} = &1-  \nonumber \\
    \frac{1}{\binom{4352/C}{N}}&\sum_{x_4=x_{4b}}^{x_{4e}}{\sum_{x_0=x_{0b}}^{x_{0e}}{\sum_{x_1=x_{1b}}^{x_{1e}}{ \sum_{x_2=x_{2b}}^{x_{2e}}{f(x_4)\prod_{i=0}^{3}{g(x_i,52)}}}}},
\end{align}

%\begin{equation}
%    %\scriptsize
%    P_{delay}^{128} = 1- \frac{1}{\binom{68}{N}}\sum_{x_4=\max(0,N-64)}^{\min(4,N)}{\sum_{x_0=\max(0,N-x_4-48)}^{\min(16,N-x_4)}{\sum_{x_1=\max(0,N-x_4-x_0-32)}^{\min(16,N-x_4-x_0)}{ \sum_{x_2=\max(0,N-x_4-x_0-x_1-16)}^{min(16,N-x_4-x_0-x_1)}{\binom{4}{x_4}(1-\frac{x_4}{4})^{16}\prod_{i=0}^{3}{\binom{16}{x_i}(1-\frac{x_i}{16})^{36}}}}}},
%\end{equation}
%%for AES-192 is
%\begin{equation}
%    %\scriptsize
%    P_{delay}^{192} = 1- \frac{1}{\binom{68}{N}}\sum_{x_4=\max(0,N-64)}^{\min(4,N)}{\sum_{x_0=\max(0,N-x_4-48)}^{\min(16,N-x_4)}{\sum_{x_1=\max(0,N-x_4-x_0-32)}^{\min(16,N-x_4-x_0)}{ \sum_{x_2=\max(0,N-x_4-x_0-x_1-16)}^{min(16,N-x_4-x_0-x_1)}{\binom{4}{x_4}(1-\frac{x_4}{4})^{16}\prod_{i=0}^{3}{\binom{16}{x_i}(1-\frac{x_i}{16})^{44}}}}}},
%\end{equation}
%%for AES-256 is
%\begin{equation}
%    %\scriptsize
%    P_{delay}^{256} = 1- \frac{1}{\binom{68}{N}}\sum_{x_4=\max(0,N-64)}^{\min(4,N)}{\sum_{x_0=\max(0,N-x_4-48)}^{\min(16,N-x_4)}{\sum_{x_1=\max(0,N-x_4-x_0-32)}^{\min(16,N-x_4-x_0)}{ \sum_{x_2=\max(0,N-x_4-x_0-x_1-16)}^{min(16,N-x_4-x_0-x_1)}{\binom{4}{x_4}(1-\frac{x_4}{4})^{16}\prod_{i=0}^{3}{\binom{16}{x_0}(1-\frac{x_i}{16})^{52}}}}}}.
%\end{equation}

where
  \begin{eqnarray*}
  % \nonumber to remove numbering (before each equation)
    &&f(x) = \binom{256/C}{x}(1-\frac{xC}{256})^{16} \\
    &&g(x,y) = \binom{1024/C}{x}(1-\frac{xC}{1024})^{y} \\
    &&x_0+x_1+x_2+x_3+x_4 = N \\
    &&x_{4b} = \max(0,N-4096/C) \\
    &&x_{4e} = \min(256/C,N) \\
    &&x_{0b} = \max(0,N-x_4-3072/C) \\
    &&x_{0e} = \min(1024/C,N-x_4) \\
    &&x_{1b} = \max(0,N-x_4-x_0-2048/C) \\
    &&x_{1e} = \min(1024/C,N-x_4-x_0) \\
    &&x_{2b} = \max(0,N-x_4-x_0-x_1-1024/C) \\
    &&x_{2e} = \min(1024/C,N-x_4-x_0-x_1)
  \end{eqnarray*}

When the AES implementation uses 5KB lookup tables, the $P_{delay}$ for AES-128, AES-192 and AES-256 are as follows respectively:
\begin{align}
    P_{delay}^{128} = &1-   \nonumber \\
    \frac{1}{\binom{5120/C}{N}}&\sum_{x_4=x_{4b}}^{x_{4e}}{\sum_{x_0=x_{0b}}^{x_{0e}}{\sum_{x_1=x_{1b}}^{x_{1e}}{ \sum_{x_2=x_{2b}}^{x_{2e}}{f(x_4,16)\prod_{i=0}^{3}{f(x_i,36)}}}}},
\end{align}
%for AES-192 is
\begin{align}
    %\scriptsize
    P_{delay}^{192} = &1-  \nonumber \\
    \frac{1}{\binom{5120/C}{N}}&\sum_{x_4=x_{4b}}^{x_{4e}}{\sum_{x_0=x_{0b}}^{x_{0e}}{\sum_{x_1=x_{1b}}^{x_{1e}}{ \sum_{x_2=x_{2b}}^{x_{2e}}{f(x_4,16)\prod_{i=0}^{3}{f(x_i,44)}}}}},
\end{align}
%for AES-256 is
\begin{align}
    %\scriptsize
    P_{delay}^{256} = &1-  \nonumber \\
    \frac{1}{\binom{5120/C}{N}}&\sum_{x_4=x_{4b}}^{x_{4e}}{\sum_{x_0=x_{0b}}^{x_{0e}}{\sum_{x_1=x_{1b}}^{x_{1e}}{ \sum_{x_2=x_{2b}}^{x_{2e}}{f(x_4,16)\prod_{i=0}^{3}{f(x_i,52)}}}}},
\end{align}
where
  \begin{eqnarray*}
  % \nonumber to remove numbering (before each equation)
    &&f(x,y) = \binom{1024/C}{x}(1-\frac{xC}{1024})^{y} \\
    &&x_0+x_1+x_2+x_3+x_4 = N \\
    &&x_{4b} = \max(0,N-4096/C) \\
    &&x_{4e} = \min(1024/C,N) \\
    &&x_{0b} = \max(0,N-x_4-3072/C) \\
    &&x_{0e} = \min(1024/C,N-x_4) \\
    &&x_{1b} = \max(0,N-x_4-x_0-2048/C) \\
    &&x_{1e} = \min(1024/C,N-x_4-x_0) \\
    &&x_{2b} = \max(0,N-x_4-x_0-x_1-1024/C) \\
    &&x_{2e} = \min(1024/C,N-x_4-x_0-x_1)
  \end{eqnarray*}

%\begin{equation}
%    %\scriptsize
%    P_{delay}^{128} = 1- \frac{1}{\binom{80}{N}}\sum_{x_4=\max(0,N-64)}^{\min(16,N)}{\sum_{x_0=\max(0,N-x_4-48)}^{\min(16,N-x_4)}{\sum_{x_1=\max(0,N-x_4-x_0-32)}^{\min(16,N-x_4-x_0)}{ \sum_{x2=\max(0,N-x_4-x_0-x_1-16)}^{min(16,N-x_4-x_0-x_1)}{\binom{16}{x_4}\binom{16}{x_0}\binom{16}{x_1}\binom{16}{x_2}\binom{16}{x_3}} (1-\frac{x_4}{16})^{16}\prod_{i=0}^{3}{(1-\frac{x_i}{16})^{36}}}}},
%\end{equation}
%%for AES-192 is
%\begin{equation}
%    %\scriptsize
%    P_{delay}^{192} = 1- \frac{1}{\binom{80}{N}}\sum_{x_4=\max(0,N-64)}^{\min(16,N)}{\sum_{x_0=\max(0,N-x_4-48)}^{\min(16,N-x_4)}{\sum_{x_1=\max(0,N-x_4-x_0-32)}^{\min(16,N-x_4-x_0)}{ \sum_{x_2=\max(0,N-x_4-x_0-x_1-16)}^{min(16,N-x_4-x_0-x_1)}{\binom{16}{x_4}\binom{16}{x_0}\binom{16}{x_1}\binom{16}{x_2}\binom{16}{x_3}} (1-\frac{x_4}{16})^{16}\prod_{i=0}^{3}{(1-\frac{x_i}{16})^{44}}}}},
%\end{equation}
%%for AES-256 is
%\begin{equation}
%    %\scriptsize
%    P_{delay}^{256} = 1- \frac{1}{\binom{80}{N}}\sum_{x_4=\max(0,N-64)}^{\min(16,N)}{\sum_{x_0=\max(0,N-x_4-48)}^{\min(16,N-x_4)}{\sum_{x_1=\max(0,N-x_4-x_0-32)}^{\min(16,N-x_4-x_0)}{ \sum_{x_2=\max(0,N-x_4-x_0-x_1-16)}^{min(16,N-x_4-x_0-x_1)}{\binom{16}{x_4}\binom{16}{x_0}\binom{16}{x_1}\binom{16}{x_2}\binom{16}{x_3}} (1-\frac{x_4}{16})^{16}\prod_{i=0}^{3}{(1-\frac{x_i}{16})^{52}}}}}.
%\end{equation}
%where
%\begin{equation*}
%  x_0+x_1+x_2+x_3+x_4=N.
%\end{equation*}

When the AES implementation uses 4KB lookup tables, the $P_{delay}$ for AES-128, AES-192 and AES-256 are as follows respectively:
\begin{align}
    P_{delay}^{128} = &1-   \nonumber \\
    \frac{1}{\binom{4096/C}{N}}&\sum_{x_0=x_{0b}}^{x_{0e}}{\sum_{x_1=x_{1b}}^{x_{1e}}{\sum_{x_2=x_{2b}}^{x_{2e}}{\prod_{i=0}^{3}{f(x_i,40)}}}},
\end{align}
%for AES-192 is
\begin{align}
    %\scriptsize
    P_{delay}^{192} = &1-  \nonumber \\
    \frac{1}{\binom{4096/C}{N}}&\sum_{x_0=x_{0b}}^{x_{0e}}{\sum_{x_1=x_{1b}}^{x_{1e}}{\sum_{x_2=x_{2b}}^{x_{2e}}{\prod_{i=0}^{3}{f(x_i,48)}}}},
\end{align}
%for AES-256 is
\begin{align}
    %\scriptsize
    P_{delay}^{256} = &1-  \nonumber \\
    \frac{1}{\binom{4096/C}{N}}&\sum_{x_0=x_{0b}}^{x_{0e}}{\sum_{x_1=x_{1b}}^{x_{1e}}{\sum_{x_2=x_{2b}}^{x_{2e}}{\prod_{i=0}^{3}{f(x_i,56)}}}},
\end{align}
where
  \begin{eqnarray*}
  % \nonumber to remove numbering (before each equation)
    &&f(x,y) = \binom{1024/C}{x}(1-\frac{xC}{1024})^{y} \\
    &&x_0+x_1+x_2+x_3 = N \\
    &&x_{0b} = \max(0,N-3072/C) \\
    &&x_{0e} = \min(1024/C,N) \\
    &&x_{1b} = \max(0,N-x_0-2048/C) \\
    &&x_{1e} = \min(1024/C,N-x_0) \\
    &&x_{2b} = \max(0,N-x_0-x_1-1024/C) \\
    &&x_{2e} = \min(1024/C,N-x_0-x_1)
  \end{eqnarray*}

%\begin{equation}
%    %\scriptsize
%    P_{delay}^{128} = 1- \frac{1}{\binom{64}{N}}\sum_{x_0=\max(0,N-48)}^{\min(16,N)}{\sum_{x_1=\max(0,N-x_0-32)}^{\min(16,N-x_0)}{ \sum_{x_2=\max(0,N-x_0-x_1-16)}^{min(16,N-x_0-x_1)}{\binom{16}{x_0}\binom{16}{x_1}\binom{16}{x_2}\binom{16}{x_3}} \prod_{i=0}^{3}{(1-\frac{x_i}{16})^{40}}}},
%\end{equation}
%%for AES-192 is
%\begin{equation}
%    %\scriptsize
%    P_{delay}^{192} = 1- \frac{1}{\binom{64}{N}}\sum_{x_0=\max(0,N-48)}^{\min(16,N)}{\sum_{x_1=\max(0,N-x_0-32)}^{\min(16,N-x_0)}{ \sum_{x_2=\max(0,N-x_0-x_1-16)}^{min(16,N-x_0-x_1)}{\binom{16}{x_0}\binom{16}{x_1}\binom{16}{x_2}\binom{16}{x_3}} \prod_{i=0}^{3}{(1-\frac{x_i}{16})^{48}}}},
%\end{equation}
%%for AES-256 is
%\begin{equation}
%    %\scriptsize
%    P_{delay}^{256} = 1- \frac{1}{\binom{64}{N}}\sum_{x_0=\max(0,N-48)}^{\min(16,N)}{\sum_{x_1=\max(0,N-x_0-32)}^{\min(16,N-x_0)}{ \sum_{x_2=\max(0,N-x_0-x_1-16)}^{min(16,N-x_0-x_1)}{\binom{16}{x_0}\binom{16}{x_1}\binom{16}{x_2}\binom{16}{x_3}} \prod_{i=0}^{3}{(1-\frac{x_i}{16})^{56}}}}.
%\end{equation}
%where
%\begin{equation*}
%  x_0+x_1+x_2+x_3=N.
%\end{equation*}

When the AES implementation uses 2KB lookup table, the $P_{delay}$ for AES-128, AES-192 and AES-256 are as follows respectively:
\begin{equation}
    %\scriptsize
    P_{delay}^{128} = 1- (1-\frac{NC}{2048})^{16*10},
\end{equation}
%for AES-192 is
\begin{equation}
    %\scriptsize
    P_{delay}^{192} = 1- (1-\frac{NC}{2048})^{16*12},
\end{equation}
%for AES-256 is
\begin{equation}
    %\scriptsize
    P_{delay}^{256} = 1- (1-\frac{NC}{2048})^{16*14}.
\end{equation} 
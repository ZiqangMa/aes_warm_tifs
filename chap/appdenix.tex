\newpage
%\appendix
\section{The detail proof of Theorem~\ref{lemma:warmnum1} and Theorem~\ref{lemma:warmnum2}}
\label{appendixa}
In this section, we describe the proof details of Theorem~\ref{lemma:warmnum1} and Theorem~\ref{lemma:warmnum2}.


\subsection{The proof of Theorem~\ref{lemma:warmnum1}}
To prove the Theorem~\ref{lemma:warmnum1}, we need to compare $E(T^{L})$ with $E(T^{C})$. 
So we get the Formula~\ref{equ:tpc1}, and estimate if it is larger than $0$.
By combing Equation~\ref{equ:pic} and ~\ref{equ:pil}, we get Equation~\ref{equ:ypwarm1} and \ref{equ:detail1}.

For the calculation, $0 \leq P_{a1} \leq 1$, $0 \leq P_{a2} \leq 1$, $0 \leq P_{a2}^{'} \leq 1$, $0 \leq P_{a2}^{''} \leq 1$, $0 \leq P_{Lwarm} \leq 1$, $0 \leq P_{Mwarm} \leq 1$
 and $0 \leq P_{evict} \leq 1$.
 
%\begin{footnotesize}
\begin{equation}
\label{equ:ypwarm1}
\begin{split}
%\footnotesize
%\scriptsize
y(P_{Lwarm}) =& -P_{evict}P_{a2}(P_{a2}^{'}+P_{evict}(P_{a3}-P_{a1}))*P_{Lwarm} \\
& +  P_{evict}P_{a2}^{'}(P_{a2}+P_{evict})(1-P_{a1})\\
& - P_{evict}^{2}P_{a2}(1-P_{a3}) \,
\end{split}\end{equation}

\begin{equation}
\label{equ:detail1}
\begin{split}
E(T^{L}) - E(T^{C}) = &\frac{y(P_{Lwarm})T_{delay}}{A_{L}*B_{L}}
\end{split}\end{equation}
%\end{footnotesize}
where
\begin{equation*}
\begin{split}
&A_{L}=P_{Lwarm}P_{a2}^{'}+P_{evict}P_{Lwarm}(1-P_{a1}) \\ & \ \ \ \ \ \ \ \ + P_{evict}(1-P_{Lwarm})(1-P_{a3})+P_{evict}P_{a2}^{'}\\
&B_{L}=P_{a2}+P_{evict}P_{a2}+P_{evict}(1-P_{a1})
\end{split}\end{equation*}

For the function $y(P_{Lwarm})$, it is a monotony decrease function. When $P_{Lwarm} = 1$, it gets the minimum value. 
At this point, $P_{a2}^{'} = P_{a2}$ and $y(P_{Lwarm}) = 0$.
Also $A_{L}$ and $B_{L}$ are larger than zero.
Therefore $E(T^{L}) - E(T^{C})\geq 0$.
That means the extra time introduced by less {\vwarm} strategy is always larger than conditional {\vwarm} strategy.

\subsection{The proof of Theorem~\ref{lemma:warmnum2}}










The symmetry axis of $y(P_{warm})$ is $P_{warm} = P_{a}$, where $P_{a}$ is denoted in Equation~\ref{equ:pa}.
%\begin{footnotesize}
\begin{equation}
\label{equ:pa}
\begin{split}
P_{a} = &\frac{1}{2(P_{evict}+P_{delay})}\{k*P_{evict}P_{delay}\\
& + P_{evict}^{2}P_{delay} \\
&+ P_{evict}P_{delay}^{2} - P_{evict}^{2} \\
& - P_{evict}P_{delay}\} \ .
\end{split}
\end{equation}
%\end{footnotesize}
Equation~\ref{equ:pal0} holds when $0.8 \leq P_{delay} \leq 1$, $0 \leq P_{evict} \leq 1$,  which means $P_{a}  \geq 0$.
%\begin{footnotesize}
  \begin{equation}
  \label{equ:pal0}
  \begin{split}
     k  > 1 > & \frac{P_{evict}^{2} + P_{evict}P_{delay} - P_{evict}P_{delay}^{2} - P_{evict}^{2}P_{delay}}{P_{evict}P_{delay}} \\
     &=1+ \frac{P_{evict}}{P_{delay}} - P_{delay} - P_{evict} \ .
  \end{split}
  \end{equation}
 % \end{footnotesize}
  %We make $k_{1} = \frac{P_{evict}^{2} + P_{evict}P_{delay} - P_{evict}P_{delay}^{2} - P_{evict}^{2}P_{delay}}{P_{evict}P_{delay}}$.
  %After simplified, $k_{1} = \frac{P_{evict}}{P_{delay}} - P_{delay} + 1 - P_{evict}$.
  %$k_{1}$ is a decreasing function of $P_{delay}$, the maximum of $k_{1}$ is $0.25P_{evict}+0.2$ when $P_{delay} = 0.8$. The maximum value is less than 1, but $k$ must be larger than $1$. So this condition is false.
 %means $k<1$ which conflicts with the precondition $k>1$.



\begin{enumerate}


  \item If $P_{a} > 1$ (i.e., Equation~\ref{equ:pal1} holds), the minimum value of $y(P_{warm})$ is $y(1)$.
  When $k \epsilon (k_{1}, k_{0}]$, $y(P_{warm}) \geq 0$ holds.
  %\begin{footnotesize}
    \begin{equation}
    \label{equ:pal1}
     k_{1} = \frac{(P_{evict}+P_{delay})(2+P_{evict}-P_{delay}P_{evict})}{P_{evict}P_{delay}} \ .
  \end{equation}
  %\end{footnotesize}
  %In this condition,.
 % \begin{equation}
  %  \scriptsize
  %  y(1) = (P_{delay} - 1)P_{evict}P_{delay}k + (1 + P_{evict}- P_{evict}P_{delay})(P_{evict}+P_{delay}) \ .
 % \end{equation}
  %make $y(1) \geq 0$, so all $y(P_{warm}) \geq 0$ when $0 \leq P_{warm} \leq 1$ . So:
  %\begin{footnotesize}
  \begin{equation}
   k_{0} = \frac{(1 + P_{evict}- P_{evict}P_{delay})(P_{evict}+P_{delay})}{(1 - P_{delay})P_{evict}P_{delay}}\ .
  \end{equation}
  %\end{footnotesize}
  %We mark $k_{2} = \frac{(P_{evict}+P_{delay})(2+P_{evict}-P_{delay}P_{evict})}{P_{evict}P_{delay}}$ and \par $k_{0} = \frac{(1 + P_{evict}- P_{evict}P_{delay})(P_{evict}+P_{delay})}{(1 - P_{delay})P_{evict}P_{delay}}$ ,\par so when
  %$k_{2} < k \leq k_{0},$
 % $$y(P_{warm}) \geq 0$$.\par

 % \ \par

  \item If $0 \leq P_{a} \leq 1,$ (i.e., $1 < k \leq k_{1}$), the minimum value of $y(P_{warm})$ is $y(P_{a})$.
  %\begin{scriptsize}
  \begin{equation}\begin{split}
    %\scriptsize
    &y(P_{a}) = \frac{1}{4(P_{evict}+P_{delay})}\{-P_{evict}^{2}P_{delay}^{2}k^{2}\\
    &+ 2P_{evict}^{2}P_{delay}(2P_{delay}+P_{evict}-P_{evict}P_{delay})k\\
    &+ 2P_{evict}P_{delay}^{2}(2P_{delay}+P_{evict}-P_{evict}P_{delay})k\\
    & - P_{evict}^{2}(P_{evict}+P_{delay})^{2}(1-P_{delay})^{2}\}.
  \end{split}\end{equation}
%\end{scriptsize}
 $y(P_{a})$ is a function of $k$ (denoted as $f(k)$). As $-P_{evict}^{2}P_{delay}^{2} < 0$, the minimum of $f(k)$ is either $f(1)$ or $f(k_{1})$, for $1 < k \leq k_{1}$. When $P_{delay} \in [0.8,1]$, $f(1)$ and  $f(k_{2})$ is larger than $0$, which means $y(P_{a}) > 0$

  % So if $f(1)>0 and f(k_{2})>0$, $f(k)>0 in k \in (1,k_{2}]$.
  %\begin{scriptsize}
  \begin{equation}
  \label{equ:f1}
  \begin{split}
    %\scriptsize
    f(1) =& \frac{1}{4(P_{evict}+P_{delay})}\{-(1-P_{delay})^{2}P_{evict}^{4}\\ & + 2P_{evict}^{3}P_{delay}^{2} + 4P_{evict}P_{delay}^{3} \\ &+ P_{evict}^{2}P_{delay}^{2}(4-P_{delay}^{2} - 2P_{evict}P_{delay})\}.
  \end{split}\end{equation}%\end{scriptsize}
 % So, $f(1) > 0$ is true when $P_{delay} \in [0.8,1]$.

  %\begin{footnotesize}
  \begin{equation}
  \label{equ:fk1}
    \raggedleft
  \begin{split}
    f(k_{1}) = &(P_{evict}+P_{delay})\{2P_{delay}- 1\\ & + P_{evict}P_{delay}(1 - P_{delay})\}.
  \end{split}\end{equation}
  %\end{footnotesize}
  %So, $f(k_{2}) > 0$ is true when $P_{delay} \in [0.8,1]$.\par

 % \ \par

  %Therefore, when $1< k \leq k_{2}$, $y(P_{a}) > 0$.

  %So, $y(P_{warm}) > 0$.

\end{enumerate}
%So, we can get that when $1< k \leq k_{0}$, $y(P_{warm}) \geq 0$.
From the above analysis, we find that when $1< k \leq k_{0}$, $E(T^{p}) \geq E(T^{c})$.
%I.e. when $1< k \leq k_{0}$, $T^{p} \geq T^{C}$

\section{The $P_{delay}$ in different AES key lengthes and different AES implementations }
\label{appendixb}
We assume the size of a cache line is $C$ Byte. $P_{delay}$ represents that the probability of performing delay operation with $N$ cache line size of lookup tables not in the cache. So when the AES implementation uses 4.25KB lookup tables, the $P_{delay}$ for AES-128, AES-192 and AES-256 are as follows respectively:
\begin{align}
%\begin{flalign}
%\begin{split}
    P_{delay}^{128} = &1-   \nonumber \\
    \frac{1}{\binom{4352/C}{N}}&\sum_{x_4=x_{4b}}^{x_{4e}}{\sum_{x_0=x_{0b}}^{x_{0e}}{\sum_{x_1=x_{1b}}^{x_{1e}}{ \sum_{x_2=x_{2b}}^{x_{2e}}{f(x_4)\prod_{i=0}^{3}{g(x_i,36)}}}}},
%\end{split}
%\end{flalign}
\end{align}
%for AES-192 is
\begin{align}
    %\scriptsize
    P_{delay}^{192} = &1-  \nonumber \\
    \frac{1}{\binom{4352/C}{N}}&\sum_{x_4=x_{4b}}^{x_{4e}}{\sum_{x_0=x_{0b}}^{x_{0e}}{\sum_{x_1=x_{1b}}^{x_{1e}}{ \sum_{x_2=x_{2b}}^{x_{2e}}{f(x_4)\prod_{i=0}^{3}{g(x_i,44)}}}}},
\end{align}
%for AES-256 is
\begin{align}
    %\scriptsize
    P_{delay}^{256} = &1-  \nonumber \\
    \frac{1}{\binom{4352/C}{N}}&\sum_{x_4=x_{4b}}^{x_{4e}}{\sum_{x_0=x_{0b}}^{x_{0e}}{\sum_{x_1=x_{1b}}^{x_{1e}}{ \sum_{x_2=x_{2b}}^{x_{2e}}{f(x_4)\prod_{i=0}^{3}{g(x_i,52)}}}}},
\end{align}

%\begin{equation}
%    %\scriptsize
%    P_{delay}^{128} = 1- \frac{1}{\binom{68}{N}}\sum_{x_4=\max(0,N-64)}^{\min(4,N)}{\sum_{x_0=\max(0,N-x_4-48)}^{\min(16,N-x_4)}{\sum_{x_1=\max(0,N-x_4-x_0-32)}^{\min(16,N-x_4-x_0)}{ \sum_{x_2=\max(0,N-x_4-x_0-x_1-16)}^{min(16,N-x_4-x_0-x_1)}{\binom{4}{x_4}(1-\frac{x_4}{4})^{16}\prod_{i=0}^{3}{\binom{16}{x_i}(1-\frac{x_i}{16})^{36}}}}}},
%\end{equation}
%%for AES-192 is
%\begin{equation}
%    %\scriptsize
%    P_{delay}^{192} = 1- \frac{1}{\binom{68}{N}}\sum_{x_4=\max(0,N-64)}^{\min(4,N)}{\sum_{x_0=\max(0,N-x_4-48)}^{\min(16,N-x_4)}{\sum_{x_1=\max(0,N-x_4-x_0-32)}^{\min(16,N-x_4-x_0)}{ \sum_{x_2=\max(0,N-x_4-x_0-x_1-16)}^{min(16,N-x_4-x_0-x_1)}{\binom{4}{x_4}(1-\frac{x_4}{4})^{16}\prod_{i=0}^{3}{\binom{16}{x_i}(1-\frac{x_i}{16})^{44}}}}}},
%\end{equation}
%%for AES-256 is
%\begin{equation}
%    %\scriptsize
%    P_{delay}^{256} = 1- \frac{1}{\binom{68}{N}}\sum_{x_4=\max(0,N-64)}^{\min(4,N)}{\sum_{x_0=\max(0,N-x_4-48)}^{\min(16,N-x_4)}{\sum_{x_1=\max(0,N-x_4-x_0-32)}^{\min(16,N-x_4-x_0)}{ \sum_{x_2=\max(0,N-x_4-x_0-x_1-16)}^{min(16,N-x_4-x_0-x_1)}{\binom{4}{x_4}(1-\frac{x_4}{4})^{16}\prod_{i=0}^{3}{\binom{16}{x_0}(1-\frac{x_i}{16})^{52}}}}}}.
%\end{equation}

where
  \begin{eqnarray*}
  % \nonumber to remove numbering (before each equation)
    &&f(x) = \binom{256/C}{x}(1-\frac{xC}{256})^{16} \\
    &&g(x,y) = \binom{1024/C}{x}(1-\frac{xC}{1024})^{y} \\
    &&x_0+x_1+x_2+x_3+x_4 = N \\
    &&x_{4b} = \max(0,N-4096/C) \\
    &&x_{4e} = \min(256/C,N) \\
    &&x_{0b} = \max(0,N-x_4-3072/C) \\
    &&x_{0e} = \min(1024/C,N-x_4) \\
    &&x_{1b} = \max(0,N-x_4-x_0-2048/C) \\
    &&x_{1e} = \min(1024/C,N-x_4-x_0) \\
    &&x_{2b} = \max(0,N-x_4-x_0-x_1-1024/C) \\
    &&x_{2e} = \min(1024/C,N-x_4-x_0-x_1)
  \end{eqnarray*}

When the AES implementation uses 5KB lookup tables, the $P_{delay}$ for AES-128, AES-192 and AES-256 are as follows respectively:
\begin{align}
    P_{delay}^{128} = &1-   \nonumber \\
    \frac{1}{\binom{5120/C}{N}}&\sum_{x_4=x_{4b}}^{x_{4e}}{\sum_{x_0=x_{0b}}^{x_{0e}}{\sum_{x_1=x_{1b}}^{x_{1e}}{ \sum_{x_2=x_{2b}}^{x_{2e}}{f(x_4,16)\prod_{i=0}^{3}{f(x_i,36)}}}}},
\end{align}
%for AES-192 is
\begin{align}
    %\scriptsize
    P_{delay}^{192} = &1-  \nonumber \\
    \frac{1}{\binom{5120/C}{N}}&\sum_{x_4=x_{4b}}^{x_{4e}}{\sum_{x_0=x_{0b}}^{x_{0e}}{\sum_{x_1=x_{1b}}^{x_{1e}}{ \sum_{x_2=x_{2b}}^{x_{2e}}{f(x_4,16)\prod_{i=0}^{3}{f(x_i,44)}}}}},
\end{align}
%for AES-256 is
\begin{align}
    %\scriptsize
    P_{delay}^{256} = &1-  \nonumber \\
    \frac{1}{\binom{5120/C}{N}}&\sum_{x_4=x_{4b}}^{x_{4e}}{\sum_{x_0=x_{0b}}^{x_{0e}}{\sum_{x_1=x_{1b}}^{x_{1e}}{ \sum_{x_2=x_{2b}}^{x_{2e}}{f(x_4,16)\prod_{i=0}^{3}{f(x_i,52)}}}}},
\end{align}
where
  \begin{eqnarray*}
  % \nonumber to remove numbering (before each equation)
    &&f(x,y) = \binom{1024/C}{x}(1-\frac{xC}{1024})^{y} \\
    &&x_0+x_1+x_2+x_3+x_4 = N \\
    &&x_{4b} = \max(0,N-4096/C) \\
    &&x_{4e} = \min(1024/C,N) \\
    &&x_{0b} = \max(0,N-x_4-3072/C) \\
    &&x_{0e} = \min(1024/C,N-x_4) \\
    &&x_{1b} = \max(0,N-x_4-x_0-2048/C) \\
    &&x_{1e} = \min(1024/C,N-x_4-x_0) \\
    &&x_{2b} = \max(0,N-x_4-x_0-x_1-1024/C) \\
    &&x_{2e} = \min(1024/C,N-x_4-x_0-x_1)
  \end{eqnarray*}

%\begin{equation}
%    %\scriptsize
%    P_{delay}^{128} = 1- \frac{1}{\binom{80}{N}}\sum_{x_4=\max(0,N-64)}^{\min(16,N)}{\sum_{x_0=\max(0,N-x_4-48)}^{\min(16,N-x_4)}{\sum_{x_1=\max(0,N-x_4-x_0-32)}^{\min(16,N-x_4-x_0)}{ \sum_{x2=\max(0,N-x_4-x_0-x_1-16)}^{min(16,N-x_4-x_0-x_1)}{\binom{16}{x_4}\binom{16}{x_0}\binom{16}{x_1}\binom{16}{x_2}\binom{16}{x_3}} (1-\frac{x_4}{16})^{16}\prod_{i=0}^{3}{(1-\frac{x_i}{16})^{36}}}}},
%\end{equation}
%%for AES-192 is
%\begin{equation}
%    %\scriptsize
%    P_{delay}^{192} = 1- \frac{1}{\binom{80}{N}}\sum_{x_4=\max(0,N-64)}^{\min(16,N)}{\sum_{x_0=\max(0,N-x_4-48)}^{\min(16,N-x_4)}{\sum_{x_1=\max(0,N-x_4-x_0-32)}^{\min(16,N-x_4-x_0)}{ \sum_{x_2=\max(0,N-x_4-x_0-x_1-16)}^{min(16,N-x_4-x_0-x_1)}{\binom{16}{x_4}\binom{16}{x_0}\binom{16}{x_1}\binom{16}{x_2}\binom{16}{x_3}} (1-\frac{x_4}{16})^{16}\prod_{i=0}^{3}{(1-\frac{x_i}{16})^{44}}}}},
%\end{equation}
%%for AES-256 is
%\begin{equation}
%    %\scriptsize
%    P_{delay}^{256} = 1- \frac{1}{\binom{80}{N}}\sum_{x_4=\max(0,N-64)}^{\min(16,N)}{\sum_{x_0=\max(0,N-x_4-48)}^{\min(16,N-x_4)}{\sum_{x_1=\max(0,N-x_4-x_0-32)}^{\min(16,N-x_4-x_0)}{ \sum_{x_2=\max(0,N-x_4-x_0-x_1-16)}^{min(16,N-x_4-x_0-x_1)}{\binom{16}{x_4}\binom{16}{x_0}\binom{16}{x_1}\binom{16}{x_2}\binom{16}{x_3}} (1-\frac{x_4}{16})^{16}\prod_{i=0}^{3}{(1-\frac{x_i}{16})^{52}}}}}.
%\end{equation}
%where
%\begin{equation*}
%  x_0+x_1+x_2+x_3+x_4=N.
%\end{equation*}

When the AES implementation uses 4KB lookup tables, the $P_{delay}$ for AES-128, AES-192 and AES-256 are as follows respectively:
\begin{align}
    P_{delay}^{128} = &1-   \nonumber \\
    \frac{1}{\binom{4096/C}{N}}&\sum_{x_0=x_{0b}}^{x_{0e}}{\sum_{x_1=x_{1b}}^{x_{1e}}{\sum_{x_2=x_{2b}}^{x_{2e}}{\prod_{i=0}^{3}{f(x_i,40)}}}},
\end{align}
%for AES-192 is
\begin{align}
    %\scriptsize
    P_{delay}^{192} = &1-  \nonumber \\
    \frac{1}{\binom{4096/C}{N}}&\sum_{x_0=x_{0b}}^{x_{0e}}{\sum_{x_1=x_{1b}}^{x_{1e}}{\sum_{x_2=x_{2b}}^{x_{2e}}{\prod_{i=0}^{3}{f(x_i,48)}}}},
\end{align}
%for AES-256 is
\begin{align}
    %\scriptsize
    P_{delay}^{256} = &1-  \nonumber \\
    \frac{1}{\binom{4096/C}{N}}&\sum_{x_0=x_{0b}}^{x_{0e}}{\sum_{x_1=x_{1b}}^{x_{1e}}{\sum_{x_2=x_{2b}}^{x_{2e}}{\prod_{i=0}^{3}{f(x_i,56)}}}},
\end{align}
where
  \begin{eqnarray*}
  % \nonumber to remove numbering (before each equation)
    &&f(x,y) = \binom{1024/C}{x}(1-\frac{xC}{1024})^{y} \\
    &&x_0+x_1+x_2+x_3 = N \\
    &&x_{0b} = \max(0,N-3072/C) \\
    &&x_{0e} = \min(1024/C,N) \\
    &&x_{1b} = \max(0,N-x_0-2048/C) \\
    &&x_{1e} = \min(1024/C,N-x_0) \\
    &&x_{2b} = \max(0,N-x_0-x_1-1024/C) \\
    &&x_{2e} = \min(1024/C,N-x_0-x_1)
  \end{eqnarray*}

%\begin{equation}
%    %\scriptsize
%    P_{delay}^{128} = 1- \frac{1}{\binom{64}{N}}\sum_{x_0=\max(0,N-48)}^{\min(16,N)}{\sum_{x_1=\max(0,N-x_0-32)}^{\min(16,N-x_0)}{ \sum_{x_2=\max(0,N-x_0-x_1-16)}^{min(16,N-x_0-x_1)}{\binom{16}{x_0}\binom{16}{x_1}\binom{16}{x_2}\binom{16}{x_3}} \prod_{i=0}^{3}{(1-\frac{x_i}{16})^{40}}}},
%\end{equation}
%%for AES-192 is
%\begin{equation}
%    %\scriptsize
%    P_{delay}^{192} = 1- \frac{1}{\binom{64}{N}}\sum_{x_0=\max(0,N-48)}^{\min(16,N)}{\sum_{x_1=\max(0,N-x_0-32)}^{\min(16,N-x_0)}{ \sum_{x_2=\max(0,N-x_0-x_1-16)}^{min(16,N-x_0-x_1)}{\binom{16}{x_0}\binom{16}{x_1}\binom{16}{x_2}\binom{16}{x_3}} \prod_{i=0}^{3}{(1-\frac{x_i}{16})^{48}}}},
%\end{equation}
%%for AES-256 is
%\begin{equation}
%    %\scriptsize
%    P_{delay}^{256} = 1- \frac{1}{\binom{64}{N}}\sum_{x_0=\max(0,N-48)}^{\min(16,N)}{\sum_{x_1=\max(0,N-x_0-32)}^{\min(16,N-x_0)}{ \sum_{x_2=\max(0,N-x_0-x_1-16)}^{min(16,N-x_0-x_1)}{\binom{16}{x_0}\binom{16}{x_1}\binom{16}{x_2}\binom{16}{x_3}} \prod_{i=0}^{3}{(1-\frac{x_i}{16})^{56}}}}.
%\end{equation}
%where
%\begin{equation*}
%  x_0+x_1+x_2+x_3=N.
%\end{equation*}

When the AES implementation uses 2KB lookup table, the $P_{delay}$ for AES-128, AES-192 and AES-256 are as follows respectively:
\begin{equation}
    %\scriptsize
    P_{delay}^{128} = 1- (1-\frac{NC}{2048})^{16*10},
\end{equation}
%for AES-192 is
\begin{equation}
    %\scriptsize
    P_{delay}^{192} = 1- (1-\frac{NC}{2048})^{16*12},
\end{equation}
%for AES-256 is
\begin{equation}
    %\scriptsize
    P_{delay}^{256} = 1- (1-\frac{NC}{2048})^{16*14}.
\end{equation} 